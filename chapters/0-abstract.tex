\renewcommand*\abstractname{Abstract}

\begin{abstract}    
Chemical modelling is essential to comprehend many environmental problems.
We present a flexible and general computer software project to calculate the geochemical modelling speciation dynamically and efficiently. 
These applications require an extremely high level of computations in a single simulation. 
The method uses the stoichiometric approach (also known as law of mass-action approach) coupled with mass-action equations and a system of equilibrium constants solved using Newton's method. An adaptive control scheme of internal factors of the simulation is adopted to guarantee that chemical and physical processes are respected and follow the literature. 
The software operates with customised options as capabilities to specify pH (activity of Hydrogen ions) of an aqueous solution, concentrations of the species, convergence criteria and maximal number of iterations. The proposed algorithm developed is described carefully in a software development point of view. 
We present a comparison study about the existing geochemical modelling solvers. Finally, a study case using different solvers applied to the same system: evaporites (halite and sylvite) in an aqueous solution containing sodium (\ce{Na^+}), chlorine (\ce{Cl^-}) and potassium (\ce{K^+}). An application of this geological behaviour is the influence of the salt dome in turbidite reservoirs of oil and gas.
\end{abstract}
