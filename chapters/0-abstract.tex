\renewcommand*\abstractname{Abstract}

\begin{abstract}    
%Chemical modelling is essential to comprehend many environmental problems.
%We present a flexible and general computer software, called \emph{SHPECK}, to calculate the geochemical speciation modelling dynamically and efficiently. 
%Geochemical modelling applications require an extremely high level of computations in a single simulation. 
%The method uses the stoichiometric approach (also known as law of mass-action approach) coupled with mass-action equations and a system of equilibrium constants solved using Newton's method. An adaptive control scheme of internal factors of the simulation is adopted to guarantee that chemical and physical processes are respected and follow the literature. 
%The software operates with customised options as capabilities to specify pH (activity of Hydrogen ions) of an aqueous solution, concentrations of the species, convergence criteria and maximal number of iterations. The proposed algorithm developed is described carefully from a software development point of view. 
%We present a comparison study about the existing geochemical modelling solvers. Finally, a study case using different solvers applied to the same system: evaporites (halite and sylvite) in an aqueous solution containing sodium (\ce{Na^+}), chlorine (\ce{Cl^-}) and potassium (\ce{K^+}). An application of this geological behaviour is the influence of the salt dome in turbidite reservoirs of oil and gas.
A geochemical speciation modelling software is responsible for calculating the distribution of dissolved species between solutes and aqueous complexes, and also computes saturation indexes for different minerals. In this work we introduce \emph{SHPECK}, a software developed to model geochemical equilibrium systems using the mass-balance conditions based on the phase rule concept \cite{Garrels:65}. 
\emph{SHPECK} composes a system of mass-action equations coupled with equilibrium constraints and solve using \emph{Newton-Raphson} method. Our software accepts any general combination of elements, species, and reactions, allowing the user to create different environments, simulations and, therefore, fully control any aspect and configuration of the model. It provides an interactive and intuitive user interface as well as the support of a built-from-the-ground database structure that handles the management of the whole thermodynamic data used for the geochemical modeling. 
Also, we present the basic concepts for geochemical modeling followed by a computer science based review about the available geochemical modeling software.
Finally, we validate \emph{SHPECK} by modeling the diagenetic reactions observed in a sandstones' reservoir and performing a comparative study with the previously discussed software. In addition to this, a database comparison was addressed and the results demonstrate a substantial improvement on the performance by the use of the \emph{SHPECK}'s relational database.
\end{abstract}



