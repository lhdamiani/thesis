%%%%%%%%%%%
%                            %
% CONCLUSION   %
%                            %
%%%%%%%%%%%

\chapter{Conclusion}

Geochemical speciation is critical for understanding the form of chemicals of interest in natural systems. It is crucial in many different aspects of our daily life nowadays: assessing bioavailability, risk to humans and ecosystems. Geochemical speciation models are generaly determined through analytical methods that measure free ions or total concentrations, used in conjunction with thermodynamically based models. As presented on this work, these models rely on the local equilibrium assumption, and on experimentally determined reaction constants. It is clear, though, that if the chemical system does not achieve equilibrium state or if the equilibrium constants are not certain, the geochemical speciation model's prediction show significant uncertainty.
This emphasizes the clear need for a proper computation approach while treating with such influential information like a geochemical speciation model. The information flow is tightly connected and any mistake will propagate errors and carry that wrong information until the very end of the model.

As computer simulation methods have gained importance in more and more disciplines, the issue of their trustworthiness for generating new knowledge has grown; therefore, we present an interesting study case where is possible to absorb and realize the comprehensiveness of this work.

Calculation of speciation is conducted by substitution of the equilibrium constants into the mass balance expressions for the total concentration of a particular component. This results in a series of nonlinear equations which are solved iteratively using numerical techniques, such as the Newton-Raphson iteration used in the MINTEQA2 model. Iterations continue until the total calculated component concentrations, derived from the equilibrium expressions, calculated activity coefficients, and component mass balances, and the measured total concentrations converge to a prescribed limit.

\begin{itemize}
\item Review of the results emphasizing our approach and advantages
\item  Geological explanations for the results
\item  Final discussion about \emph{SHPECK}
\item  CONTRIBUTION (EXPLICIT AND WELL DESCRIBED) 
\end{itemize}


%This work provides three important steps of the development of a geochemical modeling speciation software with the purpose of increasing the quality in such field of softwares. The development of a new software is intended to raise and optimize in every possible way the concept of geochemical modelling always taking into account the computer science priorities and theory. The difference between all the options and this work's approach is that the computer science's side is taken more responsibly and strategically thought since the beginning. What have been notested from all existing softwares studied is that they were created by geochemists, chemicals, geologists to solve their daily challenge and questions - our approach is different: it is done by computer science specialists with the support of geologists and geochemicals. Naturally some challenges are faced, for example, the base concepts are sometimes misunderstood and it is mandatory that the basic concepts are well stablished in order to have a solid and useful software in the end of the process. During the process of developing this work aspects of physical chemistry, aqueous geochemistry, linear algebra, complex coupled systems and computational simulatosrs are going to be studied and learned.