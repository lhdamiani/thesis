%%%%%%%%%%%
%                            %
% CONCLUSION   %
%                            %
%%%%%%%%%%%

\chapter{Conclusion}
\label{chapter:conclusion}

This work provides details of \emph{SHPECK} that has the purpose of increasing the quality of geochemical speciation modelling software. \emph{SHPECK} is intended to raise the quality of geochemical speciation modelling software by including an interactive and intuitive user interface as well as a relational database as nowadays software require. It is also an important goal to implement a dynamic and useful tool which is ready to be introduced into any geochemical situation or operate as an academic didactic tool for students  to help the understanding of geochemical speciation modelling software. Besides, it must be also mentioned here that \emph{SHPECK} accepts any general combination of elements, species and reactions. \emph{SHPECK} allows the user to create different environments, simulations, and, therefore,  fully control any aspect and configuration of the model.

\emph{SHPECK} has the purpose of raise and optimize in every possible way the concept of geochemical modelling, always taking into account the computer science's priorities, theories and power when applied to multidisciplinary domains. As would be expected, the first part of this work was to understand the area: geochemical modelling and geochemistry concepts. After that, we could study, use and analyze the available options to perform geochemical simulations. At this moment, the foundations for creating \emph{SHPECK} were already solid. We could, therefore, clearly see that there was a colossal need for a solution that would bring either computer science and geochemical simulations side-by-side into the geochemical modelling \emph{'world'}.

What has been noticed from all existing software studied is that they were created by geochemists to solve their daily challenge and questions. Our approach is different: computer science specialists do it with the support of geologists and geochemists.

The first challenge was to design the database that would fit the information of the geochemical speciation modelling software. This information was parsed from a well-stablished source and organised in a \emph{SQLite} relational database that provides fast access to information and powerful queries. Once \emph{SHPECK}'s database was ready to boost it with the data, the speciation technique itself was now on the light spot. Speciation is the calculation of the distribution of dissolved species between free ions and aqueous complexes and saturation indexes for different minerals. \emph{SHPECK} is a geochemical speciation modelling software that calculates the speciation of a solution based on a set of mass-balance equations, which are solved iteratively using the Newton-Raphson's method.
Once the implementation of both database and mathematical treatment was effectively finished, the new objective, was to implement a spontaneous, interactive and intuitive interface. \emph{SHPECK}'s \emph{GUI} allows interaction seamlessly and smoothly - this way the user can focus on what matter: modelling the geochemical environment.

After the software is adequately concluded, the new target, as important as its implementation, is to test, analyze, compare and measure the quality and accuracy of the results provided. Possibly at this stage of the project any problem could appear that was undistinguished until this point of the work.

Fortunately, the results are positive when a series of tests hit both database, algorithm and mathematical accuracy.
\emph{SHPECK}'s database has improved approximately 40\% in the average time elapsed to fetch the information if compared to the regular flat file database used by others.
\emph{SHPECK}'s results accuracy is equivalent and inside the expect range to the study case developed in this work. The results show that all the three software compared have nearly the same results if with the same inputs. Slight differences are found when simulating with temperatures higher than $100^o$C - which are acceptable and totally understandable taking into account the thermodynamic properties of the equilibrium constant.

Due to reasons, like time and purpose, there are still some points along this work where there are enhancements to be done. For instance, add the kinetic reactions to \emph{SHPECK}'s solver - it is a  delicate topic since adding kinetic reactions implies not only working with mass balance equations but also with chemical elemental mass evolution. The elemental mass evolution describes mass change through mass transfer and kinetic reactions of solids and solute-solute interaction - described in details in \cite{Ajpark:14}.
\newpage

%Summary here? Maybe it also doesn't make sense?
%\begin{itemize}
%\item Review of the results emphasizing our approach and advantages
%\item  Geological explanations for the results
%\item  Final discussion about \emph{SHPECK}
%\item  CONTRIBUTION (EXPLICIT AND WELL DESCRIBED) 
%\end{itemize}


