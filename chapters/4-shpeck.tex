%%%%%%%%%%%%%%%%%%%
%                                                        %
% SHPECK - SPECIATION MODEL  %
%                                                        %
%%%%%%%%%%%%%%%%%%%

\chapter{\emph{SHPECK} - Geochemical Speciation Model}
\label{chapter:SHPECK}

\emph{SHPECK} is a geochemical speciation model based on the \emph{Phase Rule} described in \cite{Garrels:65}. This is also the principle that is used on the speciation models reviewed in the previous section. 

As a computer simulation software, \emph{SHPECK} is a complete integrated model that allows: the user to set up choosing; convert the chemical description to a system of equations; solving the system of equations; and present the results.

Figure~\ref{fig:shpeck_flow} represents a briefing overview to make it clear what are the processes involved in \emph{SHPECK}. In figure~\ref{fig:shpeck_flow} ($a$), the user will interact with the interface to define the speciation settings and the chemical phases that compose the present water (necessary information will come from the database but is transparent to the user). In figure~\ref{fig:shpeck_flow} ($b$), the processing core will take place and perform the several tasks involved in defining the speciation of a solution - described in details later in this chapter. In figure~\ref{fig:shpeck_flow} ($c$) , the relevant results are displayed in the \emph{GUI} to the user.

\begin{figure}[ht!]
\centering
\includegraphics[width=100mm]{figures/shpeck_flow.png}
\caption{\emph{SHPECK} overview}
\label{fig:shpeck_flow}
\end{figure}

This chapter will guide through the aspects of the development of \emph{SHPECK} - a geochemical speciation modeling software. Software quality factors (either external and internal) were constantly addressed and studied. As external factors, we can mention:
\begin{itemize}
\item Correctness: The ability of performing the exact task as defined by the specification of a geochemical modeling software.
\item Robustness: The ability of reacting appropriately to abnormal conditions that may be entered through the chemical composition by the user.
\item Extendibility: The ease of adapting to changes of specification and reorganize the tasks.
\item Reusability: The ability to serve for construction of many different applications. For example, adopting \ce{CO_2} pressure or including reactive transport to the solver.
\item Compatibility: The ease of combining software elements with others. Working with new datasets and trying to connect different source of thermodynamic data can be complicated - if not impossible.
\end{itemize}
On the other side, we have internal factors:
\begin{itemize}
\item Modularity: By dividing the system into different parts is possible to have a better understanding. A set of different modules compose the whole system - each one responsible for a task.
\item Readability: Adopting code conventions and good habits in programming are fundamental to the health of the software.
\end{itemize}
These quality factors are detailed in \cite{Meyer:00}.

\section{Specification}
\emph{SHPECK} is a geochemical speciation modeling software designed to calculate the distribution of dissolved chemical elements and aqueous solutes and complexes, and it also calculates saturation indexes for different minerals. 

\section{Documentation}
For internal documentation, \emph{Doxygen} is used. \emph{Doxygen} is an automated code documentation generating tool \cite{Heesch:13}. This documentation will be available along with the distribution of \emph{SHPECK} when possible.

\section{Intended users}
\emph{SHÈCK} is a software for geochemists, geologists, chemists, environmental engineers, forestry engineers, agronomists, laboratory analysts and others. 

\section{Architecture}
The architecture of the software is responsible for assuring that all the internal and external stakeholders' concerns are preserved, addressed and satisfied. The design of the architecture must occur prior the start of the development - it avoids waste of resources and re-work in the future. \emph{SHPECK}'s architectural design choices are done always to make sure that the software will bring value to the user. 


Figure~\ref{fig:shpeck-architecture} shows the software architecture of \emph{SHPECK}, which is modeled following the popular concept of \emph{Model-View-Controller} (MVC) \cite{Gamma:94}. \emph{MVC} is an architectural pattern that divides the software into three interconnected parts:
\begin{itemize}
\item Model: It is an object representing data or even activity. For example, the algorithm and math behind calculating the activity coefficient by Debye-Hueckels' formula.
\item View: It is a form of visualization of the state of the model. For example, which are the solutes that the user wants to add into the simulation?
\item Controller: It offers facilities to change the state of the model. For example, define which algorithm for calculating the activity coefficient according to the user's choice or the value of the ionic strength (if the user did not specify which one to use).
\end{itemize}
In figure~\ref{fig:shpeck-architecture} any kind of information flow is indicated by the direction of the arrows going out or in from the components.

\begin{figure}[ht!]
\centering
\includegraphics[width=100mm]{figures/shpeck-architecture.png}
\caption{Architecture of the software \emph{SHPECK}}
\label{fig:shpeck-architecture}
\end{figure}

\subsection{Technical Specification}
\emph{SHPECK} is a software developed using C++, which is a general-purpose programming language. C++ has a bias toward system programming that supports efficient low-level computation, data abstraction, object-oriented programming, and generic programming - as shown in \cite{Dale:04} and \cite{Stroustrup:97}. It provides powerful and flexible mechanisms for abstraction. In other words, the language allows the programmer to introduce and use new types of objects that match the concepts of an application needed. Thus, C++ supports styles of programming that rely on relatively direct manipulation of hardware resources to deliver a high degree of efficiency. It can also address higher-level styles of programming that rely on user-defined types to provide a model of computation that is closer to human's view of the task being performed by a computer. Application libraries often support these higher level styles of programming. While developing Shpeck, two supporter tools were adopted:
\begin{itemize}
\item \emph{Qt} : It is a cross-platform application and UI framework for developers using C++  \cite{Qt:14}. \emph{Qt} turns the development fast and easy. With extensive built-in library classes available, it provides a comprehensive range of functionality, as well as a convenient platform.
\item \emph{Armadillo} : It is a C++ linear algebra library that provides classes for vectors, matrices, cubes and a whole set of functions to operate on the classes \cite{arma:14}. Its syntax is similar to Matlab, and it has an up-to-date support with upgrades and new releases are available on monthly basis.
\end{itemize}
Important to mention that the system is being developed using \emph{MAC} but since the beginning of the development process, the goal is to develop a multi-platform software to be distributed either in \emph{Windows} and \emph{LINUX-like} systems.

~\ref{Meyer:00} concepts and arguments are maintained close during the development of \emph{SHPECK} to produce o high-class software.

\section{Governing equations}
%These chemical equations represent the mathematical model that drives the methods that are executed in \emph{SHPECK}. Their behavior defines the steps and the order in which they are presented. This set of equations models a real-world system and is named \emph{computer simulation model}.

\emph{SHPECK} uses thermodynamic equilibrium reactions as equations for the calculation of multiphase systems in equilibrium. Details and treatments of these reactions are discussed in details in chapter ~\ref{chapter:basic}. A set of mass-action equations (as in equation ~\ref{eq:massaction}) compose the system, and the number of species and compounds that coexist in the system defines the number of equations. These equations model the geochemical speciation in a closed system taking into account the chemical properties of the solutes.

Aside from the mass-action equations there must be additional constraints to solve the equilibrium state of the system. In \emph{SHPECK}, I use the concentration of the species to deal with the equilibrium state of the system. Therefore, I have the following configuration:

\begin{equation}\label{eq:configuration}
S_{aqueous} = N_{reactions} + N_{eq constraints}
\end{equation}
where $S_{aqueous}$ are the number of aqueous solutes in the solution, $N_{reactions}$ are the number of mass-action equations and $N_{eq constraints}$ are the number of equilibrium constraints imposed by the user. 
In order to have a better efficiency on the method, we use equation ~\ref{eq:massactionLog} reformulated by applying natural logarithm in both sides, as expressed below:
\begin{equation}\label{eq:massactionLog}
ln({K_j}) =  ln({\prod\limits_{i=1}^N  a_i^{v_{ij}}}) \hspace{35pt}    (j = 1, ... , M)
\end{equation}

\section{Numerical Method}
In order to solve the system composed of the equilibrium state of the mass-action equations and the equilibrium constraints, \emph{SHPECK} uses a numerical method that solves the complete set of equations simultaneously and find the value of the unknowns (solute concentrations).


The numerical method applied is a modification of the \emph{Newton's method} (also known as Newton-Raphson method), which is a method for finding successively better approximations to the roots of the given set of equations. It works with a derivative approach to the equations, which optimizes the time consumed to find the roots and makes the representation of the system easier.

\emph{SHPECK} uses the \emph{Newton-Raphson} method to solve a nonlinear system of equations which results in finding the roots of continuously differentiable equations. 

The \emph{Newton-Raphson} method is applied in order to achieve the best approximation possible to the solution that is being sought. It is an iterative algorithm where each step consists in minimizing the first-order approximation of the solution and its difference from regular \emph{Newton}'s method is that second derivatives are not required. \emph{Newton-Raphson}'s method is used to solve a system of coupled nonlinear equations. The first-order approximation of the function starts with an initial guess for the minimum values, the method proceeds by the iterations as shown in equations~\ref{eq:NewtonMethodEq1}.

\begin{equation}
\label{eq:NewtonMethodEq1}
F(x+1) = F(x) - J^{-1} * R
\end{equation}
where $F$ is function's result for the applied $x$, $J^{-1}$ is the inverse of the Jacobian matrix, and $R$ is the residual vector (\cite{Isaacson:66}). The \emph{Newton-Raphson}'s method has a complexity O($n^3$) per iteration with a quadratic convergence.

The \emph{Jacobian Matrix} is the matrix of all first-order partial derivatives of the equations and it is defined as

\begin{equation}
\label{eq:JacobianDefinition}
J_{mn} = \frac{\partial y^m}{\partial x^n}
\end{equation}
where the $y^i$'s are a new coordinate system defined in terms of the original coordinate system, the $x^i$'s. In differential equation theory, the Jacobian matrix plays a key role in defining the stability of solutions.

The residual vector ($R$) is defined as a vector containing the resulting values for each equation.

\begin{equation}
\label{eq:residualVector}
R = \begin{pmatrix}
 F(x_1) \\
 \vdots \\
 F(x_m)
 \end{pmatrix}
\end{equation}
where $m$ is the number of unknowns (or mass-action equations plus equilibrium constraints).

The algorithm consists of iteratively calculating new approximations for the unknown values, through the matrix equation:

\begin{equation}
\label{eq:iterativelyAlgorithm}
[J]^{-1}_{iteration}* \alpha [U]_{iteration+1} = [R]_{iteration}
\end{equation}
where $J$ is the Jacobian Matrix; $ \alpha [U]_{iteration+1} $ is the unknown composition at the next iteration; iteration is the iteration number and $R$ is the residual vector. With this, it is possible to state the $[U]_{iteration+1}$ value with:

\begin{equation}
\label{eq:CompositionCalculation}
[U]_{iteration+1} = [U]_{iteration} + \alpha [U]_{iteration+1}
\end{equation}

The initial guess of the solutions is an approximation that the user provides to the \emph{Newton-Raphson} method. This method needs a \emph{seed} to start the calculations (usually this guess is used for $F(0)$). 
If the guess is close to the real root value the number of iterations necessary to obtain the solution is small. If the guess is far from the real solution, more iterations are needed to find the correct solution.

Specifically, \emph{SHPECK}'s equations can be described as: $F_1(x_1,..., x_n)$,...,$F_m(x_1,...,x_n)$. The partial derivatives of all these equations with respect to the variables $x_1,...,x_n$ can be organized in a m-by-n matrix, the Jacobian matrix, as bellow:

\begin{equation} 
J =
 \begin{pmatrix}
  \frac{\partial F_1}{\partial x_1} & \cdots & \frac{\partial F_1}{\partial x_n} \\
  \vdots  & \ddots & \vdots  \\
  \frac{\partial F_m}{\partial x_1} & \cdots &   \frac{\partial F_m}{\partial x_n}
 \end{pmatrix}
\end{equation}

In our case, $m = n$ and the \emph{jacobian matrix} is a square matrix generated once and updated every iteration. With the equations selected and organized, the derivatives of each reaction towards each variable are calculated and the \emph{jacobian matrix} is modeled and stored. 

It is important to realize that the main complication of using the \emph{Newton-Raphson} method to solve a system of nonlinear equations is defining the equations (and their derivatives) included in the \emph{Jacobian Matrix}. 
As the number of equations and unknowns increases ($n$), so does the number of elements in the \emph{Jacobian} ($n^2$).
\section{Algorithm}

\emph{SHPECK}'s algorithm takes as its input a specification of the system's state. For example, this consists of the values for the concentrations of the solutes in the solution, the temperature of the system, the method used to calculate activity coefficient, etc. It then calculates the system's state by executing the numerical iterative process of achieving a solution. 
Figure~\ref{fig:Shpeck-algo} presents the high-level algorithm in details. It is important to understand that each of the boxes in this algorithm represents a set of instructions, calculations, and conditional clauses. Due to this work's scope we only specify the algorithm inside the box called "Newton's Method Solver". The algorithm of how the governing equations control the numerical methods is presented in figure~\ref{fig:Shpeck-algo-newton}. 
\begin{figure}[ht!]
\centering
\includegraphics[width=100mm]{figures/Shpeck_algo3.png}
\caption{High-level algorithm of \emph{SHPECK}}
\label{fig:Shpeck-algo}
\end{figure}
\begin{figure}[ht!]
\centering
\includegraphics[width=100mm]{figures/Shpeck_algo_newton.png}
\caption{Algorithm of the Newton's Method Solver}
\label{fig:Shpeck-algo-newton}
\end{figure}

%(http://ranger.uta.edu/~huber/cse4345/Notes/Nonlinear_Equations.pdf)
%(http://ranger.uta.edu/~huber/cse4345/Notes/Equations.pdf)
%(compared to http://en.citizendium.org/wiki/Newton%27s_method#Computational_complexity)


\section{Graphical User Interface}
Due to the enormous amount of options connected to the nature of geochemical modeling, it is necessary to develop the software as an intuitive and user-friendly utility. The development of the GUI introduces several stages of processes related to it:
\begin{itemize}
\item Planning : This moment is where the developer needs to conceptualize and evaluate the options toward constructing the core of the program that is intuitive and friendly to the user. The planning is an essential process. The content, features, and details of the software need to be strictly defined and organized. There is a small gap from an useful software from one that is not useful at all. Especially, guessing the role of a user is a difficult task, simply because the developer is not the final user for its applications. It is important also to find out if there are comparable software already in the public domain.
\item Building : The planning for the implementation of the features of the software is important to be well defined at this point. To build the tool \emph{Qt} - already presented above - was chosen due to its many advantages, since it is easy to customize, no coding was necessary, templates are available, provides a simple drag and drop \emph{GUI}  builder, and is secure and reliable;
\item Ensuring Usability : The challenge in the field of Human-Computer Interaction (HCI) investigates the way in which people use computers. Techniques and tools are used to find relevant standards. These include tests with real users, evaluation by experts, gathering user feedback, and usage logging;
\end{itemize}


\subsection{\emph{SHPECK}'s GUI}
In existing geochemical modeling software systems, the \emph{GUI} are either poorly implemented, or not implemented at all. The \emph{GUI} of \emph{SHPECK} enables the user to streamline and conveniently construct a geochemical environment for modeling. \emph{SHPECK}'s \emph{GUI} works based on tabs, where each tab is responsible for viewing a specific point of the software. It is presented them in detail below:
\begin{itemize}
\item Configurations tab: this allows users to view and manipulate basic system settings and to control temperature, activity coefficient calculation method, the number of iterations, solver options, and error and convergence criteria. It is presented in  figure~\ref{fig:config}.
\item Chemical phases in the water tab: it allows users to create and edit the composition of the water that will be used in the geochemical speciation model. It has the complete catalog of species available from the database. The species in the tab that have concentrations other than zero compose the water. This tab is presented in figure~\ref{fig:water}.
\item Tab of results: Allows the user to see relevant input information and the outputs of the geochemical speciation model, such as, temperature, ionic strength, pH of the solution, final concentration for the species, saturation indexes, etc. It is presented  figure~\ref{fig:output}.
\end{itemize}


\begin{figure}[ht!]
\centering
\includegraphics[width=100mm]{figures/shpeck-configtab.png}
\caption{Configuration tab}
\label{fig:config}
\end{figure}

\begin{figure}[ht!]
\centering
\includegraphics[width=100mm]{figures/shpeck-watertab.png}
\caption{Chemical phases in the water tab}
\label{fig:water}
\end{figure}

\begin{figure}[ht!]
\centering
\includegraphics[width=100mm]{figures/shpeck-resultstab.png}
\caption{Results tab}
\label{fig:output}
\end{figure}


\subsubsection{GUI Interaction}
To demonstrate how the user interaction works in \emph{SHPECK}, the selection of which method used to calculate the activity coefficient will be detailed. Figure~\ref{fig:radioAct} shows the available options (displayed to the user by radio buttons). There are four: automated, \emph{Debye Hueckel}, \emph{Bdot} and \emph{Davies}. The automated option will choose among the available methods which one fits best to the modeled system. The methods were described in ~\ref{actCoeff}. Inside the logic of the program, the system identifies which one should be used (code~\ref{cod:actCoeff}) and calls the referring method (code~\ref{cod:debyeHueckelMethod}). 

\begin{figure}[ht!]
\centering
\includegraphics[width=100mm]{figures/radioAct.png}
\caption{Activity Coefficient Selection Method}
\label{fig:radioAct}
\end{figure}


\begin{minipage}{0.8\linewidth}
\begin{lstlisting}[frame=single, label=cod:actCoeff, caption=Activity coefficient calculation method implementation]
if ((iStrength < I_STRENGTH_DBH_MAX && option == 0) || (option == 1)){
    // Debie-Huckel Equations
    qDebug() << MODEL_OF_EQUATIONS_DBH;
    if (option == 0)
        selectedEqMethod = "Automatic (Debie-Huckel)";
    actCoeffList = ActCoeffDBH::actCoeffDBH(actCoeffDatabase.getADH(),
                     actCoeffDatabase.getBDH(),iStrength,
                     variables,charges,ionSizes);

\end{lstlisting}
\end{minipage}

\begin{minipage}{0.8\linewidth}
\begin{lstlisting}[frame=single, label=cod:debyeHueckelMethod, caption=Debye Hueckel method implementation]
QList<double> ActCoeffDBH::calcActCoeff(double adh,
                                        double bdh,
                                        double ionicStrength,
                                        QStringList variables,
                                        QStringList charges,
                                        QStringList ionSize)
{
    QList<double> actCoeff;
    for (int i = 0; i < variables.size(); i++){
        double value = (adh * qPow(charges.at(i).toDouble(),2.0) * qPow(ionicStrength,0.5) * -1)/
                (1 +  ionSize.at(i).toDouble()* bdh * qPow(ionicStrength,0.5));
        value = qPow(10.0,value);
        actCoeff.push_back(value);
    }
    return actCoeff;
}

\end{lstlisting}
\end{minipage}

\subsubsection{Chemical phases in the water}
The chemical composition displayed to the user in the 'compounds tab' comes from the database. It shows the chemical species that are present in \emph{SHPECK}'s database. Among these, the user can define the concentration for each one and this is corresponding to the water composition that will be simulated and speciated. A simple water composition is shown in details in figure~\ref{fig:waterCompositionDetail}. When the \emph{SHPECK} runs, this water is fetched from the database (code~\ref{cod:waterMethod}), each query is treated as one component of the water (code~\ref{cod:waterMethodSaving}) and each component is added to the water composition of the system (code~\ref{cod:waterMethodSavingComponent}).

\begin{figure}[ht!]
\centering
\includegraphics[width=50mm]{figures/shpeck-waterComposition.png}
\caption{Water composition detailed}
\label{fig:waterCompositionDetail}
\end{figure}

\begin{minipage}{0.8\linewidth}
\begin{lstlisting}[frame=single, label=cod:waterMethod, caption=Fetching the water composition from the database method]
// Get water composition information from database
status = DatabaseHelper::getWaterComposition(&waterShpeck,dataBase,&pHValue);
\end{lstlisting}
\end{minipage}

\begin{minipage}{0.8\linewidth}
\begin{lstlisting}[frame=single, label=cod:waterMethodSaving, caption=Excerpt of the water composition saving code]
    if (query.exec(sql))
    {
        while(query.next())
        {
            // Set values inside the class WaterComposition
            // from the query returned from the database
            waterShpeck->addComponent(&query);
\end{lstlisting}
\end{minipage}

\begin{minipage}{0.8\linewidth}
\begin{lstlisting}[frame=single, label=cod:waterMethodSavingComponent, caption=Detailed composition of the water]
    Composition* composition = new Composition();
    composition->setConcentration(query->value(0).toFloat());
    composition->setName(QString(query->value(1).toString()));
    composition->setCharge(query->value(2).toFloat());
    composition->setIonSize(query->value(3).toFloat());
    composition->setMoleWt(query->value(4).toFloat());
    composition->setNumberOfElements(query->value(5).toInt());
    componentsList.push_back(*composition);
\end{lstlisting}
\end{minipage}

\section{Database}
As can be seen in this chapter, the algorithm of \emph{SHPECK} is not trivial and requires many interactions between many different entities. The database is responsible for providing the data that will flow through \emph{SHPECK}. Chapter ~\ref{chapter:review} makes clear that using a flat file database poses difficulties to users. Potential issues of using flat file databases are: duplication of the information; non-unique records; difficulty in updating and maintenance; inherently inefficient; rigid (difficult data format); and insecure \ref{dauerer2000system}. 
In \emph{SHPECK}, we use a relational database, which is a model of a database that prevents the problems faced in flat file database previously discussed. A relational database works based on a \emph{query}, which is an information request from the database. Besides that, there are three important advantages:
\begin{itemize}
\item Memory saving, since the information is not fully load in main memory during run time. The software fetches the information when required by each module. The data is permanently stored and relationally structured according to the requirements of the application.
\item Direct access to relevant information: Complex queries are built and fetched when required by the software modules, avoiding sequential search over text files.  Only the chemical phases that are part of the simulated reaction are load to memory to be processed by the software. Complex queries and concatenation of queries result in a faster and more efficient use of the available resources. The semantic connection between the relational data allows richer queries semantically when fetching data.
\end{itemize}

The database architecture, presented in details in figure~\ref{fig:ERDiagram}, was defined after studying the algorithm and determining what would be the structure and the information requested.  The relational database embraces all the important data requested for a geochemical modelling software and the structure was defined in order to compact, organize, structure and make the information access efficient by the modules. 

\begin{figure}[ht!]
\centering
\includegraphics[width=100mm]{figures/ER_diagram.png}
\caption{ER Diagram of the database}
\label{fig:ERDiagram}
\end{figure}

\subsection{Database Technologies}
The implementation was done with \emph{SQLite} database \cite{SQLite}, which is a software-library that implements a self-contained, transactional \emph{SQL} open source database engine. Currently it is the most widely deployed \emph{SQL} database engine in the world. Some of its advantages are presented bellow:
\begin{itemize}
\item Zero-Configuration: \emph{SQLite} does not need to be installed before it is used, there is no setup procedure;
\item Serverless: The process that wants to acess the database reads and writes directly from the database files on disk. There is no intermediary server process (nor interprocess communication using \emph{TCP/IP});
\item Single Database File: An \emph{SQLite} database is a single file located in the directory hierarchy. \emph{SQLite} database can be easily copied onto a USB memory stick or emailed for sharing;
\item Stable Cross-Platform Database File: A database file written on one machine can be copied and used on a different machine with a different architecture. Furthermore, \emph{SQLite} is backwards compatible (newer versions can read and write older database files);
\end{itemize}

\subsection{\emph{LLNL} thermodynamic dataset parser}
Once the structure and the technology of our database were defined, a parser was prepared to extract the information from the flat file database. 
The parser was generated carefully in order to ensure that all of the data and delimiters were identified and treated properly. This task was extremely time-consuming because of irregularities found in the flat files (i.e. wrong tab or space formating, different encoding of the files, etc). Figure ~\ref{fig:parserAlgorithm} describes in details the parser.

\begin{figure}[ht!]
\centering
\includegraphics[width=100mm]{figures/parser.png}
\caption{Parser algorithm}
\label{fig:parserAlgorithm}
\end{figure}

\newpage

\section{Summary}
\begin{itemize}
\item Architecture: \emph{SHPECK} follows the architectural pattern called \emph{MVC}; Its benefits are the complete separation of responsibilities and concerns among the parts of the software. Thus, there is no mixing of codes between them. Another advantage is its flexibility that allows the software to grow and develop further. Figure ~\ref{fig:shpeck-architecture} displays the \emph{MVC} pattern existing in \emph{SHPECK}.
\item Governing Equations: \emph{SHPECK} is a geochemical speciation modeling software that drives the behavior of the aqueous system based on a set of mass-action equations combined with equilibrium constraints. A mass-action equation is described in equation  ~\ref{eq:massaction} and reinforced here:

\begin{equation}
K_j =  \prod\limits_{i=1}^N  a_i^{v_{ij}} \hspace{35pt}    (j = 1, ... , M)
\end{equation}
where $K_j$ denotes the equilibrium constant of the \emph{j-th} reaction; $a$ denotes the activity of the \emph{i-th} chemical species.

\item Numerical Method: \emph{SHPECK} applies the \emph{Newton-Raphson} method to solve the nonlinear system of equations. The concept of the method is described in equation ~\ref{eq:NewtonMethodEq1} and reinforced here:

\begin{equation}
F(x+1) = F(x) - J^{-1} * R
\end{equation}
where $F$ is function's result for the applied $x$, $J^{-1}$ is the inverse of the Jacobian matrix, and $R$ is the residual vector. The quadratic rate of convergence of this method compensate the expensive calculation inherent to it.

\item Graphical User Interface: The \emph{GUI} mission is to enable the user to use the full potential \emph{SHPECK} has to offer. It is intuitive and user-friendly to allow the user to focus on essential duties that are imperative to modeling a geochemical environment. We use the popular approach of \emph{tab} panels, which are separated according to their purposes: configurations and settings, water composition, and results visualization.
\item Database: A geochemical modeling software is dependent on the information provided in the database. \emph{SHPECK} structures its data in a \emph{SQLite} relational database, which is unique approach among the comparable software available. \emph{SHPECK}'s database is composed of the information on elements, species, compounds, reactions and thermodynamic constants found in the \emph{LLNL} thermodynamic dataset. A parser for \emph{LLNL} flat file database was created to fetch this information.
\item \emph{SHPECK}'s software engineering: A set of techniques provided in ~\ref{Meyer:00} were carefully followed along every step in the development of \emph{SHPECK}.
\item The updated version of table~\ref{tab:comparativeSoftwareTable}, including \emph{SHPECK}, is demonstrated in table ~\ref{tab:comparativeSoftwareTableUpdated}.
\begin{table}
\caption{Updated version of table ~\ref{tab:comparativeSoftwareTable}}
\label{tab:comparativeSoftwareTableUpdated}
\centering
\begin{tabular}{r|ccccccccc|ccc}

SOFTWARE &
\rot{Costs} &
\rot{Setup and versioning} & 
\rot{Customization and Integration} &
\rot{Security and Control} &
\rot{Infrastructure} &
\rot{Core functionality} &
\rot{Graphical User Interface} &
\rot{Support and Maintenance} &
\rot{Database}  &
\rot{Overall Average} 
    \\ \hline
EQ3/6        	& 2 & 2 & 1 & 1 & 3 & 5 & 1 & 1 & 1 & 1.88 \\ 
PHREEQC         & 4 & 4 & 2 & 2 & 3 & 4 & 2 & 3 & 1 & 2.77 \\ 
MINTEQA2        & 4 & 2 & 1 & 1 & 2 & 2 & 1 & 1 & 1 & 1.66\\ 
SOLMINEQ.88	    & 3 & 1 & 1 & 1 & 1 & 5 & 1 & 1 & 1 & 1.66\\ 
SHPECK	        & 4 & 4 & 2 & 3 & 3 & 5 & 5 & 5 & 5 & 4\\ 
\hline
\end{tabular}
\end{table}

\end{itemize}